
\documentclass[a4paper, 11pt]{article}
\usepackage{lipsum} %This package just generates Lorem Ipsum filler text. 
\usepackage{fullpage} % changes the margin
\usepackage{mathpazo}
\usepackage{multicol}
\usepackage{enumerate, eucal}
\usepackage{listings}
\usepackage{xcolor}
\usepackage{amsmath,amsfonts,amsthm, amssymb} % Math packages

\begin{document}
%Header-Make sure you update this information!!!!
\noindent
\large \textbf{Homework 1} \hfill \textbf{Xun Gong(517020910141} \\
\normalsize {\bf CS 259 @ SJTU} \hfill ACM Class, Zhiyuan College, SJTU\\
Prof.~{\bf David Bindel} \hfill Due Date: May 29, 2019\\
TA.~{\bf Zhou Fan} \hfill Submit Date: \today

\section*{Problem 1: Constrained least squares}

\subsection*{(1)}

\begin{align*}
&\because \sum x_i = 1 \\
&\therefore || x ||^2 = 1 \\
&\mathcal{L} = ||Ax - b||^2 + \lambda ||x||^2 + \mu A^T (b - Ax) \\
&\delta \mathcal{L} = \delta ||Ax - b||^2 + \lambda * \delta ||x||^2 + \delta \mu A^T (b - Ax)\\
&\delta \mathcal{L} = 2*\delta x^T (A^T A x - A^T b - \lambda x) = 0\\
\end{align*}

$$\therefore KKT 
\begin{cases}
    x^* = (A^T A - \lambda)^{-1} A^T b \\
    ||x^*|| = 1 \\
\end{cases}$$

\subsection*{(2)}

\lstset{
    language=python,
    numbers=left, 
    numberstyle= \tiny, 
    keywordstyle= \color{ blue!70},
    commentstyle= \color{red!50!green!50!blue!50}, 
    rulesepcolor= \color{ red!20!green!20!blue!20} ,
    escapeinside=``, % 英文分号中可写入中文
    xleftmargin=2em,xrightmargin=2em, aboveskip=1em,
    framexleftmargin=2em
}
\begin{lstlisting}
import numpy as np
from sympy import *
q, r = np.linalg.qr(A)
p = Symbol('p')
X = (A.T*A - p).I * A.T * b
solve(np.norm(X) - 1)
\end{lstlisting}


\section*{Problem 2: Residual sensitivity}

\subsection*{(1)}
Equal to show  $||r|| \delta ||r|| = r^T \delta r $ \\
Equal to show $\delta (||r||^2) = 2 r^T \delta r $ 

\begin{align*}
    \delta (||r||^2) &= \delta (r^T r) \\
    &= (\delta r^T) r + r^T \delta r \\
    &= 2 r^T \delta r \\
\end{align*}

\subsection*{(2)}

Equal to show $||r|| \delta ||r|| = - r^T \delta A x $ \\
And from (1), $r^T \delta r = - r^T \delta A x $ \\
Equal to show $ \delta r = - \delta A x $ ()\\
And $r = b - A x$ \\
$\therefore \delta r = 0 - \delta A x $ is equal to (*).

$\square$

\end{document}
