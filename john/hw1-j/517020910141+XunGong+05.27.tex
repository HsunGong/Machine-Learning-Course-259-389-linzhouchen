\documentclass[paper=a4, fontsize=11pt]{scrartcl} % A4 paper and 11pt font size
\usepackage{algorithmic}
% \usepackage{xeCJK}
\usepackage{listings}
\usepackage{color}
\usepackage{enumitem} 
\usepackage{gensymb}
\usepackage[titletoc]{appendix}

\usepackage{fontspec}

\usepackage[T1]{fontenc} % Use 8-bit encoding that has 256 glyphs
\usepackage{fourier} % Use the Adobe Utopia font for the document - comment this line to return to the LaTeX default
\usepackage[english]{babel} % English language/hyphenation
\usepackage{amsmath,amsfonts,amsthm} % Math packages

\usepackage{lipsum} % Used for inserting dummy 'Lorem ipsum' text into the template
\usepackage{enumerate}
\usepackage{sectsty} % Allows customizing section commands
\allsectionsfont{\normalfont\scshape} % Make all sections centered, the default font and small caps

\usepackage{fancyhdr} % Custom headers and footers
\pagestyle{fancyplain} % Makes all pages in the document conform to the custom headers and footers
\fancyhead{} % No page header - if you want one, create it in the same way as the footers below
\fancyfoot[L]{} % Empty left footer
\fancyfoot[C]{} % Empty center footer
\fancyfoot[R]{\thepage} % Page numbering for right footer
\renewcommand{\headrulewidth}{0pt} % Remove header underlines
\renewcommand{\footrulewidth}{0pt} % Remove footer underlines
\setlength{\headheight}{13.6pt} % Customize the height of the header
\numberwithin{equation}{section} % Number equations within sections (i.e. 1.1, 1.2, 2.1, 2.2 instead of 1, 2, 3, 4)
\numberwithin{figure}{section} % Number figures within sections (i.e. 1.1, 1.2, 2.1, 2.2 instead of 1, 2, 3, 4)
\numberwithin{table}{section} % Number tables within sections (i.e. 1.1, 1.2, 2.1, 2.2 instead of 1, 2, 3, 4)

\setlength\parindent{2em} % Removes all indentation from paragraphs - comment this line for an assignment with lots of text

%----------------------------------------------------------------------------------------
%	TITLE SECTION
%----------------------------------------------------------------------------------------

\newcommand{\horrule}[1]{\rule{\linewidth}{#1}} % Create horizontal rule command with 1 argument of height

\title{	
\normalfont \normalsize 
\textsc{Zhiyuan College, Shanghai Jiaotong University} \\ % Your university, school and/or department name(s)
\horrule{0.5pt} \\[0.4cm] % Thin top horizontal rule
\huge CS389: Foundations of Data Science Homework I\\ % The assignment title
\horrule{2pt} \\ % Thick bottom horizontal rule
}

\author{Xun Gong, 517020910141} % Your name

\date{\normalsize\today} % Today's date or a custom date


\begin{document}

\maketitle % Print the title

\section*{Exercise 2.14}
\subsection*{(1)}
$$ 2^{d-k}C(k, d) $$
Let $Q(k, d)$ denote the number of k-dimensional faces in a d-cube.
STEP1: There are $n$ edges connected from each vertex, and we get a k-face for any subset of $k$ distinct edges from among these $d$ edges. Therefore the number of k-faces at each vertex of an d-cube is $C(k, d) = d!/[k!(d - k)!]$.
STEP2: For a n-cube, we have $2n$ surfaces. Since we have $C(k, d)$ k-cubes at each of the $2n$ surfaces, we obtain a total number $2^dC(k, d)$. 
STEP3: But in this count, each face is counted $2^k$ times, so we divide by that number to get the final formula: $Q(k, d) = 2^{n-k}C(k, d)$.

\subsection*{(2)}
$$3^d$$
A cube subdivided into smaller faces. From what shows above, 
$Q(0,d) + Q(1,d) + ... + Q(d-1,d) + Q(d,d) 
= 2^d + C(1,d)2^{d-1} + C(2,d)2^{d-2} + ... + C(d-1,d)2 + C(d,d)
= (2 + 1)^d = 3^d$

\subsection*{(3)}
$$2d$$
$Q(d-1, d) = 2d$, side-length = 1, therefore $\text{area} = 2d$


\subsection*{(4)}
$$ 2^d\cdot d $$
Area = $ 2^{d-1} \cdot 2d = 2^d\cdot d$
\subsection*{(5)}
Proof: (Use method of Section 2.3)
Let $\theta$ be the width of surface.
$$V(\textit{surface}) / V = 1 - (1-\theta)^d \geq 1 - e^{-\theta d}$$
When $d$ is large enough, no matter how small $\theta$ is, this ratio shall be close to $1$.


\section*{Exercise 2.16}

$r = 1, h = 1$

\subsection*{3-dimension}
horizontal plane:
$$ Area = Volume(2-plane) = \pi$$
half of a circular ball: 
$$ Area = \frac{1}{2} Area(3-ball) = 2\pi $$

\subsection*{4-dimension}
horizontal plane:
$$ Area = \frac{4}{3} \pi$$
half of a circular ball: 
$$ Area = \frac{1}{2} \pi^2 $$


\section*{Exercise 2.18}

Use Lemma 2.6, $$ V(d) = \int_{r=0}^{r_0} r^{d-1} dr \int_{S^d} \Omega = \frac{r^d}{d}A(d) $$
and, $$ A(d) = \frac{2 \pi}{\tau(d/2)} $$

\subsection*{r = 2}

$$ V_2(d) = \frac{2 \pi}{d*\tau(d/2)} * 2^d $$

\subsection*{r > 2, ind}

$$ V_r(d) = \frac{2 \pi}{d*\tau(d/2)} * r^d $$

\subsection*{r = f(d)}
Want $V_r(d) \rightarrow Constant as d \rightarrow \infty$, therefore,
$ lim_{d \rightarrow \infty} (v_r - C) = 0 $

$$ r = \sqrt[d]{ C*\frac{d \tau(d/2}{2 \pi} } $$
and $n! \approx (n / e)^n$
$$ r =  \sqrt{\frac{d}{2 \pi e}} \sqrt[d]{C \pi d} $$


\section*{Exercise 2.26}

Equally to prove $A \cap B = Volume$, $A$ = narrow slice at the equator, $B$ = narrow annulus at the surface.

And So, 
$$ V(A \cap B) / V = \frac{1}{1 + \frac{2}{c}e^{-c^2/2}}(1 - (1-\epsilon)^d) = 1 $$
while $d \rightarrow \infty$

therefore, they are simultaneous. 
\begin{flushright} \item{$\square$} \end{flushright}

\section*{Exercise 2.38}

Code is in the attachment.
\newline
Solution:

\begin{tabular}{cc}

k & percent of $\sqrt{k}$ (100\%) \\
100 & 8.2 \\
50 & 12.0 \\
10 & 25.8 \\ 
5 & 34.5 \\
4 & 36.2 \\
3 & 45.4 \\
2 & 60.3 \\
1 & 72.8 \\

\end{tabular}

\section*{Exercise 2.40}

Code is in the attachment.

\begin{description}
    \item Method 1: 0.6392458491507906
    \item Method 2: 0.6477522979226946
\end{description}

\noindent{Method 2 is better.}


% \appendix
% \renewcommand{\appendixname}{Appendix~\Alph{section}}

% \section{Appendix 1: Code}
\mint{python}
\begin{lstlisting}

distance = []
for i in range(num):
    for j in range(i+1, num):
        distance.append(np.linalg.norm(s[i] - s[j]))

\end{lstlisting}



\end{document}