
\documentclass[a4paper, 11pt]{article}
\usepackage{lipsum} %This package just generates Lorem Ipsum filler text. 
\usepackage{fullpage} % changes the margin
\usepackage{mathpazo}
\usepackage{multicol}
\usepackage{enumerate, eucal}
\usepackage{listings}
\usepackage{xcolor}
\usepackage{amsmath,amsfonts,amsthm, amssymb} % Math packages

\author{Xun Gong(517020910141}
\title{Homework 2}

\begin{document}
%Header-Make sure you update this information!!!!
\maketitle

\noindent
\normalsize {\bf CS389: Foundations of Data Science @ SJTU} \hfill ACM Class, Zhiyuan College, SJTU\\
Prof.~{\bf John Hopcroft} \hfill Due Date: June 03, 2019\\

\section*{Exercise 3.12}

\subsection*{(1)}

$$||A_k||_F^2 = \sum_{i=1}^k A v_i v_i^T = \sum_{i=1}^k \sigma_i^2  $$


\subsection*{(2)}

$$||A_k||_2^2 = max|A_k x| = \sigma_1 $$

\subsection*{(3)}

$$||A - A_k||_F^2 = \sum_{i=k+1}^n A v_i v_i^T = \sum_{i=k+1}^n \sigma_i^2 $$

\subsection*{(4)}
From Lemma 3.8, 
$$||A - A_k||_2^2 = \sigma_{k+1} $$

\section*{Exercise 3.13}


\begin{align*}
    & \because   A \text{ is symmetric} \\
    & \therefore A^T = A \\
    & \therefore A^2 = A^T A = V\Sigma^2 V^T = V \Sigma V^T V \Sigma V^T \\
    & \therefore A = V \Sigma V^T \text{, V is unique.} \\
    & \text{Use the same rule, } A = U \Sigma U^T \\
    & \therefore U = V \\
    & \therefore u_i = v_i \\
\end{align*}

% \subsection*{(1) $A = VDV^T$}

% \begin{align*}
%     & \because   A \text{ is symmetric} \\
%     & \therefore A \text{ is normal} \\
%     & \therefore A \text{ is diagonal} \\
%     & \therefore A \text{ can be Eigen decomposed, } A = VDV^T \\
% \end{align*}

% \subsection*{(1) $U = V$}

\section*{Exercise 3.16}

\subsection*{(1)}

Estimate [0.00390622, 0.99999237]

\subsection*{(2)}

B = \begin{pmatrix}
    4 & 0 \\
    0 & 16 \\
\end{pmatrix}

\subsection*{(3)}


\section*{Exercise 3.18}

\section*{Exercise 3.28}

\section*{Exercise 3.32}


\end{document}
